\section{Introduction}

% What is the problem?
Today's client-server based file system metadata services have scalability
problems. It takes a lot of resources to service POSIX metadata requests so
applications perform better with dedicated metadata
servers~\cite{sevilla:sc15-mantle, ren:sc2014-indexfs} but provisioning a
metadata server for every client is expensive and complicated. This problem is
exacerbated by current trends in HPC.  Architectures are transitioning from
complex storage stacks with burst buffer, file system, object store, and tape
tiers to a more simplified stack with just a burst buffer and object
store~\cite{bent:login16-hpc-trends}; this puts more pressure on data access
instead of data movement.

% What is HPC doing?
HPC's unique requirements for file systems ({\it e.g.}, fast creates) and well
defined workloads ({\it e.g.}, workloads) make relaxing POSIX sensible. Without
the constraints of strong consistency and durability, HPC systems can now do
more client side processing with an emphasis on serverless metadata services.
One popular approach is to ``decouple the namespace", where clients lock the
subtree they want exclusive access to so other clients cannot interfere. This
delayed merge ({\it i.e.} a form of eventual consistency) and relaxed
durability improves performance and scalability by avoiding the costs of RPCs,
synchronization, false sharing, and serialization.  While the performance
benefits are obvious for these users, applications that rely on stronger
consistency or durability guarantees must be re-written or deployed on a
different system. The consistency and durability semantics for these systems is
shown in Table~\ref{table:namespaces}.

\begin{figure}[tb]
\centering
\includegraphics[width=75mm]{figures/subtree-policies.png}
\caption{Administrators can assign consistency and durability policies to
subtrees to get the benefits of some of the state-of-the-art HPC architectures.
}\label{fig:subtree-policies}
\end{figure}

\begin{table}
\begin{tabular}{ r | l | l }
              & Decoupled & Global    \\
              & Namespace & Namespace \\\hline
  Example     & BatchFS~\cite{zheng:pdsw2014-batchfs} & CephFS~\cite{weil:sc2004-dyn-metadata} \\
              & DeltaFS~\cite{zheng:pdsw2015-deltafs} & IndexFS~\cite{ren:sc2014-indexfs}      \\
  Consistency & eventual & strong     \\
  Durability  & node local & global  \\
\end{tabular}
\caption{State-of-the-art systems in HPC improve file system metadata
performance by relaxing consistency and durability
guarantees.\label{table:namespaces}}
\end{table}

% What did we do
We propose subtree policies, an interface that lets future programmers control
how the storage system manages different parts of the file system namespace.  For performance one
subtree can adopt weaker consistency semantics while another subtree can retain
the rigidity of POSIX's strong consistency. Figure~\ref{fig:subtree-policies}
shows an example setup where a single global namespace has directories for
applications designed for different, state-of-the-art HPC architectures.  We
present Cudele, a prototype programmable file system that supports different
degrees of consistency and durability by exposing mechanisms used within the
file system as a client library.  Cudele supports 3 forms of consistency
(invisible, eventual, and strong) and 3 degrees of durability (none, local, and
global) giving the administrator a wide range of policies and optimizations
that can be custom fit to an application. Our contributions: 

\begin{enumerate}

  \item a prototype that lets administrators program a range of
  consistency and durability semantics (9 permutations), allowing them to custom
  fit the storage system to the application.

  \item an API for programming consistency/durability policies and assigning
  them to subtrees in the file system namespace.

  \item a comparison of the strategies used in recently proposed research systems against
  previously unexplored metadata designs.

\end{enumerate}

% Results
Our results confirm the assertions of ``clean-state" research systems that
decouple namespaces; specifically that the technique drastically improve
performance (104\(\times\) speed up) but we go a step further by quantifying
the costs of merging updates (7\(\times\) slow down) and maintaining durability
(\(10\times\) slow down). We also show the effect of having a metadata specific
file format in systems that are based on in-memory data structures.
Section~\ref{sec:related-work} places Cudele in the context of other related
work. Section~\ref{sec:posix-overheads} quantifies the cost of POSIX
consistency and system-defined durability and
Section~\ref{sec:methodology-decoupled-namespaces} presents the Cudele
prototype and API. Section~\ref{sec:implementation} describes Cudele's
mechanisms and shows how re-using internal subsystems results in an
implementation of less than 500 lines of code. The evaluation in
Section~\ref{sec:evaluation} quantifies the overheads and performance gains of
explored and previously unexplored metadata designs.

