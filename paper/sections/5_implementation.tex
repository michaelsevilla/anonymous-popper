\section{Implementation}
\label{sec:implementation}

We use a programmable storage approach~\cite{sevilla:eurosys17} to design
CudeleFS; namely, we try to leverage components inside Ceph to inherit the
robustness and correctness of the internal subsystems. In this section, we
first describe a Ceph internal subsystem or component and then we show how we
use it in CudeleFS.

% how it works in CephFS, how I used it
\subsection{Metadata Store}
\label{sec:metadata-store}

% how it works in CephFS
In CephFS, the metadata store is a data structure that represents the file
system namespace. This data structure is stored in two places: in memory ({\it
i.e.} in the collective memory of the metadata server cluster) and as objects
in the object store. In the object store, directories and their inodes are
stored together in objects to improve the performance of scans.  The metadata
store data structure is structured as a tree of directory fragments making it
easier to read and traverse.\\

% how i used it
\noindent\textbf{Cudele}: uses the metadata store format to write objects
formatted like the journal into the object store. By writing with the same
format, the metadata servers can read and use the recovery code to materialize
the updates from a client's decoupled namespace ({\it i.e.} merge).

\subsection{Journal}
\label{sec:journal}

The journal is the second way that CephFS represents the file system namespace;
it is a log of metadata updates that can materialize the namespace when the
updates are replayed onto the metadata store. The journal is a ``pile system";
writes are fast but reads are slow because state must be reconstructed.
Specifically, reads are slow because there is more state to read, it is
unorganized, and many of the updates may be redundant.\\

\noindent\textbf{Cudele}: uses the journal subsystems within the metadata
server to replay updates onto the in-memory metadata store.  When the clients
are ready to merge their updates back into the global namespace, they pass a
binary file of  metadata updates to the metadata server. The metadata server
uses its recovery code to ``replay" the updates. 

\subsection{Journal Tool}
\label{sec:journal-tool}

The journal tool is used for disaster recovery and lets administrators view and
modify the journal. It can read the journal, export the journal as a file,
erase events, and apply updates to the metadata store.  To apply journal
updates to the metadata store, the journal tool reads the journal from object
store objects and replays the updates on the metadata store in the object
store.\\

\noindent\textbf{Cudele}: using the journal tool, clients can materialize the
journal in memory and append events to their own copy. When they are done they
export their journal to a binary file. Clients also use the journal tool to
save their updates to disk or to the object store.

\begin{listing}
\begin{minted}[frame=single,
               framesep=3mm,
               xleftmargin=21pt,fontsize=\footnotesize,
               tabsize=4]{js}
{     
    "allocated_inodes": "100000"
    "interfere_policy": "block"
    "consistency": "create"
    "durability": "local_persist"
}
\end{minted}
\caption{Implementing DeltaFS with CudeleFS.}
\label{src:deltafs}
\end{listing}

\begin{listing}
\begin{minted}[frame=single,
               framesep=3mm,
               xleftmargin=21pt,fontsize=\footnotesize,
               tabsize=4]{js}
{     
    "allocated_inodes": "100000"
    "interfere_policy": "block"
    "consistency": "create+volatile_apply"
    "durability": "local_persist"
}
\end{minted}
\caption{Implementing BatchFS with CudeleFS.}
\label{src:batchfs}
\end{listing}


% Journal import
%The journal tool imports journals from binary files stored on disk.  First the
%header of the dump is sanity checked and written into RADOS to the ``header"
%object.  The ``header" object has metadata about the journal as well as the
%locations of all the journal pointers (e.g., where the tail of the journal is,
%where we are currently trimming, etc.).  Then the journal events are cleaned
%(erasing trailing events that are not part of the header) and written as
%objects into RADOS.  Note that while the journal is in RADOS, the metadata
%servers do do not have the namespace reconstructed in memory so the metadata
%cluster will not service requests relating to the journal of imported events.
%To construct the namespace in the collective memory of the metadata servers we
%need to first construct the namespace in RADOS. The journal tool can explicitly
%do this by  applying the journal to the metadata store in RAODS. This will pull
%the objects containing journal segments and replay them on the metadata store.
%Finally, we delete the journal in RADOS and restart the metadata servers so
%they rebuild their caches.

% Journal export
%First the
%journal is scanned for the header and then journal is recovered. To recover the
%journal the ``header" object is read off disk and then objects are probed in
%order and starting from the write position saved in the header. Probing will
%update the write position if it finds objects with data in them. 

% Journal export
%When exporting a journal of events, the journal tool first scans the journal to
%check for corruption. Then it recovers the journal by reading the ``header"
%object out of RADOS.  After reading the header, the journal tool can pull
%journal segments from RADOS because it knows how many objects to pull and how
%far to seek within those objects.

%% The data structures
%The metadata that the event touches, including inodes, paths, and timestamps,
%are stored as metablobs. Each piece of metadata inside a metablob is called a
%dirlump. A dump has a section for lumps (dirfrag, dirlump), roots (dentries),
%table client transactions (tid, version), renamed directory fragments (maps,
%versions, alloc/preallocated inodes), inodes starting a truncate, inodes
%finishing a truncate, destroyed inodes, and client requests. Unfortunately for
%me (and you since you are reading this paper), this does not make any sense.
%
%Each directory fragment has an associated directory lump, which is just a bunch
%of metadata. The most interesting part of the dirlump is the fullbits array,
%which has a dentry OR inode. To walk the tree, iterate over all the dirlumps
%and then all the full bits, saving off the children and inode locations. The
%children tell us which dentry names an inode has and the inode locations map
%the parent inode to its child inode and dentry.  

\subsection{Inode Cache}
\label{sec:inode-cache}

In CephFS, the inode cache reduces the number of RPCs between clients and
metadata servers. Without contention clients can resolve metadata reads locally
thus reducing operations like \texttt{lookup()}.  For example, if a client or
metadata server is not caching the directory inode, all creates within that
directory will result in a lookup and a create request. If the directory inode
is cached then only the create needs to be sent.  The size of the inode cache
is configurable so as not to saturate the memory on the metadata server --
inodes in CephFS are about 1400
bytes\footnote{http://docs.ceph.com/docs/jewel/dev/mds\_internals/data-structures/}.
The inode cache has code for manipulating inode numbers, such as pre-allocating
inodes to clients.\\

\noindent\textbf{Cudele}: uses the internal inode cache code to allocate inodes
to clients that decouple parts of the namespace and to skip inodes used by the
client at merge time.

\subsection{Large Inodes}
\label{sec:large-inodes}

In CephFS, inodes store policies for policies, like how the file is striped
across the object store or for managing subtrees for load balancing. These
policies adhere to logical partitionings of metadata or data, like Ceph pools and file
system namespace subtrees. To implement this, the namespace data structure has
the ability to recursively apply policies to subtrees and to isolate subtrees
from each other.  \\

\noindent\textbf{Cudele}: uses the large inodes to store consistency and
durability policies in the directory inode. This approach uses the File Type
interface from the Malacology programmable store
system~\cite{sevilla:eurosys17} and it tells clients how to access the
underlying metadata. The underlying implementation stores executable code in
the inode that calls the different Cudele mechanisms. Of course, there are many
security and access control aspects of this approach but that is beyond the
scope of this paper.


% General information about the journal
%The journal segments are saved as objects in RADOS.  The journal has 4
%pointers, described in `osdc/Journaler.h`:
%
%\begin{itemize}
%  \item write position: tail of the journal; points to the current session where we are appending events
%  \item unused field: where someone is reading
%  \item expire position: old journal segments
%  \item trimmed position: where daemon is expiring old items
%\end{itemize}
%
%% How the journal tool works
%Journal segments in RADDS have a header followed by serialized log events. The
%log events are read by hopping over objects using the read offset and object
%size pulled from the journal header.  After decoding them, we can examine the
%metadata (1) about the event (e.g., type, timestamps, etc.) and (2) for inodes
%that the event touches.
%
%% The metablobs
%The metadata for inodes that the event touches are called metadata blobs and
%the ones associated with events are \textbf{unordered}; this layout makes
%writing journal updates fast but the cost is realized when reading the metadata
%blobs.  It makes sense to optimize for writing since reading only occurs on
%failures. To reconstruct the namespace for the metadata blob, the journal tool
%iterates over each metadata blob in the events and builds mappings of inodes to
%directory entries (for directories) and parent inodes to child inodes/directory
%entries.

%Reads/Writes Journal Events}

% How it works
% Why we re-use stuff
%Step 3 is the most complicated and requires understanding how the snapshot is
%materialized in memory. 
%
%\subsection{Operating on Snapshots} 
%
%Our first implementation attempted to re-create journal events using the same
%libraries that the metadata server uses. To construct a \texttt{mkdir} we tried
%to instantiate a Ceph inode and directory entry for the current file/dir and
%its parent.  This is too hard because there are too many moving parts in the
%metadata server (e.g., a mdlog class, stuff in memory, assumption that we can
%traverse up and down namespace, etc.). So when I tried add dentries and inodes
%it was trying to traverse up/down and it would almost always segfault when it
%was looking for something. These metablobs are supposed to be self container --
%the problem is I do not know what is supposed to go *inside* them. 
%
%Our second idea was to copy the metadata blog and change just what we needed.
%For example, we would save a binary dump of a generic \texttt{mkdir} event on
%disk. When the application makes a directory, this dump would be loaded and the
%fields would be changed before being written back to disk. Rather than
%traversing up and down a namespace in memory of a metadata server, we should
%traverse up and down the namespace *inside* the metadata blob. This
%implementation requires disk IO and editing the log event is non-trivial for
%two reasons:
%
%\begin{itemize}
%
%  \item methods do not edit events; they just write them
%
%  \item the metadata that the event touches (e.g., the metablob) is unorganized
%  on disk for performance -- it is trade-off for writing data faster serially and
%  reconstructing information slowly since failure is not considered the norm
%
%\end{itemize}
%
%Faced with these challenges we landed on our final implementation: load the
%snapshot into the the data structures used to examine and replay journals, edit
%those data structures, and write them out to disk as binary.

% - how it actually works
% - bugs I fixed in the last couple of commits

%The metadata objects are located with naming schemes (200.000* for journal
%objects and 1.inode for metadata storage objects). 

% How it works: socket for changing daemon's internal state (debugging, logging, behaviour)
% 1. API for putting state into the daemon dynamically
% 2. Hooks directly into daemon code so we can use any parsing functionality in there
% 3. Documentation all the tunables

% 1. read journal of updates from file
% 2. call replay (uses same code as when an metadata server comes back) on each event
% 3. skip inodes so metadata server doesn't hand out those new inodes


