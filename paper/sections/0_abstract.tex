\begin{abstract}

HPC developers debate abandoning POSIX because the synchronization and
serialization overheads of providing strong consistency and durability are too
costly -- and often uneccessary -- for their applications.  Unfortunately,
designing near-POSIX file systems excludes applications that rely on strong
consistency or durability, forcing developers to re-write their applications or
deploy them on a different system.  We present a file system and API that
allows clients to specify their consistency/durability requirements and assign
them to subtrees in the namespace, allowing administrators to optimize subtrees
within the same namespace for different workloads.  We draw conclusions about
the performance impact of unexplored consistency/durability metadata designs
and show that strong consistency can cause a 104\(\times\) slow down while merging
updates (7\(\times\) slow down) and maintaining durability (\(10\times\) slow
down) have more reasonable costs.

\end{abstract}


